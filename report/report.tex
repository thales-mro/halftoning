\documentclass[]{IEEEtran}

% Your packages go here
\usepackage[utf8]{inputenc}
\usepackage{graphicx}
\usepackage{float}
\usepackage{listings}
\usepackage{xcolor}
%listings settings
\definecolor{codegreen}{rgb}{0,0.6,0}
\definecolor{codegray}{rgb}{0.5,0.5,0.5}
\definecolor{codepurple}{rgb}{0.58,0,0.82}
\definecolor{backcolour}{rgb}{0.95,0.95,0.92}
\definecolor{codeblue}{rgb}{0,0.8,0.99}
\definecolor{codeyellow}{rgb}{0.6,0.5,0}


\lstdefinestyle{vim_like}{
  backgroundcolor=\color{backcolour},   
  commentstyle=\color{codegreen},
  keywordstyle=\color{codeyellow},
  numberstyle=\tiny\color{codegray},
  stringstyle=\color{codepurple},
  basicstyle=\ttfamily\footnotesize,
  breakatwhitespace=false,         
  breaklines=true,                 
  captionpos=b,                    
  keepspaces=true,                 
  numbers=left,                    
  numbersep=5pt,                  
  showspaces=false,                
  showstringspaces=false,
  showtabs=false,                  
  tabsize=2
}
\lstset{style=vim_like}

\markboth{MO443 Digital Image Processing}{}

\begin{document}
  \title{Project 1 - Halftoning}
  \author{Thales Oliveira (RA 148051)
    \thanks{ra148051@students.ic.unicamp.br}
  }
  \maketitle
  
  \begin{abstract}

  \end{abstract}
  
\section{Introduction} 

\section{Implemented Algorithm}
 
%  \begin{lstlisting}[language=Python, caption={Nearest neighbor interpolation function.}, label={code:nninterpolation}]
% def resize(img, shape):
%     ...
%     # build the new image with shape intended
%     result = np.zeros(shape[0]*shape[1]*3).reshape(shape[0], shape[1], 3)

%     # get y and x ratios to calculate pixels positions to take
%     y_ratio = shape[0]/img.shape[0]
%     x_ratio = shape[1]/img.shape[1]

%     for y in range(shape[0]):
%         for x in range(shape[1]):
%             result[y,x] = img[int(y/y_ratio),int(x/x_ratio)]

%     return result.astype(np.uint8)
% \end{lstlisting}

% \begin{figure}[H]
%     \centering
%     \includegraphics[width=0.4\hsize]{img/o-cv2-a.png}
%     \includegraphics[width=0.4\hsize]{img/o-nninterpo-a.png}
%     \includegraphics[width=0.4\hsize]{img/o-cv2-b.png}
%     \includegraphics[width=0.4\hsize]{img/o-nninterpo-b.png}
%     \caption{Image results from resizing. Left images: 30x30 and 700x700 images reduced with OpenCV's cubic interpolation. Right images: 30x30 and 700x700 images reduced with our nearest neighbor and Gaussian blur approach}
%     \label{fig:resize-compare}
% \end{figure}

% \begin{table}[h!]
% \centering
% \begin{center}
% \begin{tabular}{ |c|c| } 
%  \hline
%  Alphabets & Description \\
%  \hline
%  {[}'\#', '@', '\%', '=', '*', ':', '-', '.', ' '{]} & Default (required) \\ 
%  \hline
%  {[}'W', 'B', 'H', 'T', 'L', 'I', '.', ' '{]}  & Uppercase letters  \\
%  \hline
%  {[}'8', '6', '2', '1', '+', '-', '.', ' '{]} & Mathematical  \\ 
%  \hline
%  ['\}', ']', '|', '!', '"', ':', '.', ' '] & Vertical \\
%  \hline
%  ['\#', '=', '"', '\~','-', '\_', '.', ' '] & Horizontal \\
%  \hline
% \end{tabular}
%  \label{table:alphabet}
%  \caption{Alphabets used in experiments}
% \end{center}
% \end{table}

\section{Experiments}

\section{Discussion}

\section{Conclusion}


\end{document}
